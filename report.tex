% !TEX TS-program = pdflatex
% !TEX encoding = UTF-8 Unicode

% This is a simple template for a LaTeX document using the "article" class.
% See "book", "report", "letter" for other types of document.

\documentclass[11pt,twocolumn]{article} % use larger type; default would be 10pt

\usepackage[utf8]{inputenc} % set input encoding (not needed with XeLaTeX)

%%% Examples of Article customizations
% These packages are optional, depending whether you want the features they provide.
% See the LaTeX Companion or other references for full information.

%%% PAGE DIMENSIONS
\usepackage{geometry} % to change the page dimensions
\geometry{a4paper} % or letterpaper (US) or a5paper or....
% \geometry{margin=2in} % for example, change the margins to 2 inches all round
% \geometry{landscape} % set up the page for landscape
%   read geometry.pdf for detailed page layout information

\usepackage{graphicx} % support the \includegraphics command and options
\usepackage{mathtools}
\usepackage{amsmath} % need this stuff for curly brackets
% \usepackage[parfill]{parskip} % Activate to begin paragraphs with an empty line rather than an indent

%%% PACKAGES
\usepackage{booktabs} % for much better looking tables
\usepackage{array} % for better arrays (eg matrices) in maths
\usepackage{paralist} % very flexible & customisable lists (eg. enumerate/itemize, etc.)
\usepackage{verbatim} % adds environment for commenting out blocks of text & for better verbatim
\usepackage{subfig} % make it possible to include more than one captioned figure/table in a single float
\usepackage{authblk}
\usepackage{pdfpages}
\usepackage{wrapfig}
\usepackage{hyperref}
\usepackage{listings}
\lstset{frame=tb,
  language=BASH,
  aboveskip=3mm,
  belowskip=3mm,
  showstringspaces=false,
  columns=flexible,
  basicstyle={\small\ttfamily},
  numbers=none,
  numberstyle=\tiny\color{gray},
  keywordstyle=\color{blue},
  commentstyle=\color{dkgreen},
  stringstyle=\color{mauve},
  breaklines=true,
  breakatwhitespace=true
  tabsize=1
}
% These packages are all incorporated in the memoir class to one degree or another...

%%% HEADERS & FOOTERS
\usepackage{fancyhdr} % This should be set AFTER setting up the page geometry
\pagestyle{fancy} % options: empty , plain , fancy
\renewcommand{\headrulewidth}{0pt} % customise the layout...
\lhead{}\chead{}\rhead{}
\lfoot{}\cfoot{\thepage}\rfoot{}

%%% SECTION TITLE APPEARANCE
\usepackage{sectsty}
\allsectionsfont{\sffamily\mdseries\upshape} % (See the fntguide.pdf for font help)
% (This matches ConTeXt defaults)

%%% ToC (table of contents) APPEARANCE
\usepackage[nottoc,notlof,notlot]{tocbibind} % Put the bibliography in the ToC
\usepackage[titles,subfigure]{tocloft} % Alter the style of the Table of Contents
\renewcommand{\cftsecfont}{\rmfamily\mdseries\upshape}
\renewcommand{\cftsecpagefont}{\rmfamily\mdseries\upshape} % No bold!

%%% END Article customizations

\title{Software Engineering For Industry \\ AcmeTelecom - Report}
\author[1]{Wael Aljeshi\thanks{wael.aljeshi10@imperial.ac.uk}}
\author[1]{Dan Demeter\thanks{dan.demeter10@imperial.ac.uk}}
\author[1]{Razvan Rosie\thanks{razvan.rosie10@imperial.ac.uk}}
\affil[1]{Department of Computing, Imperial College London}

\begin{document}
\maketitle

\begin{abstract}
killer report...
\newline

{\bf Keywords}: Laplace transform, paralellism, other stuff.
\end{abstract}


\section{Introduction}
\section{Investigating the machine}
Investigating the machine:

The forensics team extracted some vital clues on the exterior of the machine:

\subsection{For all technical enquiries dial 0700 0036}
The phone number displayed near the top of the machine required looking up historical phone number records, and was eventually traced back to the "ORSA Journal on Computing". 
A follow up visit was made to their customer services branch at:
5521 Research Park Drive
Suite 200
Catonsville, Maryland 21228-4664
USA

The team was then able to retrieve an excerpt of a document "Numerical Inversion of Laplace Transforms of Probability Distributions" (http://www.columbia.edu/~ww2040/Fall03/LaplaceInversionJoC95.pdf)



\subsection{Cyrillic script}
Some Cyrillic script was found on the machine, believed by our Business $\&$ Strategy experts to be used for marketing the machine in the Russian market.
An image of the script was analyzed and translated electronically, yielding the rough English translation of "Fast inverter integrated functions".
It is speculated that this marketing campaign was highly unsuccessful, as the marketing text was merely a catchy phrase with no detailed description of operation FOOTNOTE
(catchy marketing phrases are clearly an inferior marketing technique, and would never work [c.f. "Just do it" and "I'm lovin' it" for examples of such unsuccessful campaigns]).
Additionally, the provided help instructions were provided in a crude mixture of English and French, probably unhelpful for the Russian operators.




\subsection{Coordinates}
2.3328700E 48.8074800N
The geographic team had no 3G signal in the basement, they were unable to access the internet in order to determine the whereabouts of the coordinates inscripted on the machine. The issue was promptly escalated to the CTO, who commissioned a purchase of two horses from a local supermarket. FOOTNOTE(Despite the falling price of petrol, it remains beyond our budget. As a plus, horses are also free to park.)

The location was found to be near the Laplace train station in France. Realising the mismatch in units due to the "inverse function" applied to the coordinates, the agents determined that the devious scheme was intended to be a very subtle hint that the machine may be performing an inverse Laplace transform. In order to minimize casualties, the horses were auctioned off at the nearby "Square du Serment de Koufra" for a profit of £7.89. This was sufficient to cover the cost of lunch.



\subsection{Portrait}
As the portrait's CMYK band was much too narrow for our forensic experts, an image was instead captured and send to the GlObal clOud based Locator E-service (G.O.O.G.L.E). The subject was identified as Leonhard Euler, "a pioneering Swiss mathematician and physicist" (according to an omni-reliable, omni-malleable, source). This may hint that the machine employs Euler's methods for its computation.



\subsection{Circuit board}
The circuit board itself was identified as a product of AMD, housing a 500 MHz AMD Geode LX800 and 256 MB DDR DRAM. Such technology was not believed to exist at the time, and it seems that great efforts have been devoted to keep this technology hidden.


\subsection{Coordinates on circuit board}
Learning from their previous experience, the geographic team decided to take a plane this time. They, of course, did not learn from their *other* mistake, and hence spent quite some time travelling instead of taking the more sensible option of employing a parallel depth first search on the earth's map. The fact was reflected appropriately in their performance review. But we digress.


After an expensive trip to France, it was determined that the site referred to "Ecole Beaumont en auge" at "Rue Pierre Simon Laplace Road". The connection was made the following week, when the team realised that the circuit was meant to employ some form of Laplace transform.




\section{Serial Implemantation}

\subsection{Parallel Implemantation}



\section{Extra Work}




\section{Conclusion}


\begin{thebibliography}{9}
\bibitem{mockito}
  \emph{Mockito}.
  http://docs.mockito.google\\code.com/hg/org/mockito/Mockito.html  


  
\end{thebibliography}




\end{document}          
