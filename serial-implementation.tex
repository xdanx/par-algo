After agreeing that the machine is in fact an inverter of $Laplace$ transforms, some time was spent investigating 
this mathematical concept. Directly, the transform can be given in the following way:
$$F(s) =\int_0^\infty \! f(t)e^{-st} \, \mathrm{d}t$$

Since we have established that we need to find the inverse transform, we notice the fact that:
$$f(t) =\int_0^\infty \! F(s)e^{st} \, \mathrm{d}t$$
\newline

To recap the data which we have gathered so far, the agents are facing the following problems:
\begin{itemize}
\item{The analytical expression of the function $F$ given by \texttt{mystery.o} is unknown. }
\item{The \texttt{mystery.o} file only allows for evaluation of the function at some points passed as inputs. }
\end{itemize}

Therefore, we have realised that we need to perform numerical inverse of $Inverse$ $Laplace$ $transform$.
The paper offered us a valuable introduction to the problem that we were facing.
The \texttt{EULER} algorithm was a starting point, and we proceed to the implementation step:

\subsection{EULER Algorithm}
The best description of the \texttt{EULER} algorithm can be summarised by the formula:
$$f(t) = \frac{10^{M/3}}{t} \sum\limits_{k=0}^{2M} {\eta_k Re( F(\frac{\beta_k}{t}) )}$$
Where:
$$\beta_k = \frac{M ln(10)}{3} + i\pi k$$
$$\eta_k = (-1)^{k} \cdot \xi_k$$
$$\xi_0 = \frac{1}{2}$$
$$\xi_k = 1 \text{, for } 1 \leq k \leq M$$
$$\xi_{2M-k} = \xi_{2M-k+1}+2^{-M}\binom{M}{k} \text{, for } 1 \leq k < M$$
$$\xi_{2M} = \frac{1}{2^M}$$
\newline

The implementation of the algorithm (see \texttt{code/euler.cpp}) follows the procedure described above and can be summarised in the following manner:
\begin{itemize}
\item{Select a size for the parameter M (for the current problem we use M as 30).}
\item{Define an array \texttt{Times} to store the arguments where of the targeted function $f$.}
\item{Construct an array of factors ($\xi$) that will act as factors in the summation of $F(\frac{\beta_k}{t})$.

	\begin{itemize}
	\item Set $xi[0] = 1/2$.
	\item Fill the elements on positions $1$ $\to$ $M+1$ with 1's.
	\item Fill the rest of the array with the factors of the binomial expansion of order $2M$.
	\end{itemize}
}
\item{Precompute the constants described by the algorithm.}
\item{For each point of time $t$, compute the sum described by the algorithm. This part of the code operates in an independent way (since we compute the sum for each point) and is suitable for a parallel implementation.}
\item{Output the vector $f_{st}$ containing the results of the evaluations.}
\end{itemize} 

Our serial implementation described the same graph as the one shown, with its global maxima pointing to the hour and the global minima pointing to the minute.


\subsection{The Dubner-Abate Algorithm}
After making sure that the \texttt{EULER} implementation works, the team decided to test other algorithms in order to compare the results and efficiency.
We present here the $Dubner \& Abate$ algorithm, which is also computes the inverse of a $Laplace$ transform as a $Fourier$ series.
The mathematics behind this algorithm is simpler because it does not require the computation of the binomial coefficients (which involve factorials) and may produce overflows.

In short, the $Dubner$-$Abate$ algorithm will compute the following:
\begin{align*}
f(t) = \frac{e^{At}}{q} \sum\limits_{k=1}^{M}{Re(L(A+t\frac{ki\pi}{q}) \cdot cos(t\frac{k\pi}{q}))} \\
- \sum\limits_{k=1}^{M}{Im(L(A+t\frac{ki\pi}{q}) \cdot sin(t\frac{k\pi}{q}) )}
\end{align*}
\newline

The serial implementation of the $Dubner$-$Abate$ algorithm follows the logic described above.
\begin{itemize}
\item{Select a size for the parameter M (for the current problem we use M as 200, for a reasonable approximation of the function $f$).}
\item{Define an array \texttt{Times} to store the arguments where $f$ will be approximated. We also precompute the constants described by the algorithm.}
\item{For each point of time $t$, compute the $Dubner$-$Abate$ algorithm's sum above.
This part of the code operates in an independent fashion and (since we compute the sum for each point) and is suitable for a parallel implementation.}
\item{Output the vector \texttt{f\_ts} containing the results of the evaluations.}
\end{itemize}
The fact that it independently computes $f(t)$  for every point makes this algorithm suitable for parallelisation.

