After agreeing that the machine is in fact an inverter of $Laplace$ transforms, some time was spent investigating 
this mathematical concept. Directly, the transform can be given in the following way:
$F(s) =\int_0^\infty \! f(t)e^{-st} \, \mathrm{d}t$.

Since we have established that we need to find the inverse transform, we notice the fact that:
$f(t) =\int_0^\infty \! F(s)e^{st} \, \mathrm{d}t$. 
\newline

To recap the date we have  gather so far, the agents are facing the following problem:
\begin{itemize}
\item{The analytical expression of the function $F$ given by $mystery.o$ is unknown. }
\item{The $mistery.o$ file only allows for evaluation of the function at some points passed as inputs. }
\end{itemize}

Therefore, we have realised that we need to perform numerical inverse of $Inverse$ $Laplace$ $transform$.
The paper offered us a valuable introduction to the problem that we were facing.
The $EULER$ algorithm was a starting point, and we proceed to the implementation step:

\subsection{EULER Algorithm}
The best description\cite{abate} of the $EULER$ algorithm can be summarised by the following formula:
$f(t) = \frac{10^{M/3}}{t} \sum\limits_{k=0}^{2M} {\eta_k Re F(\frac{\beta_k}{t}) }$, where $\beta_k = \frac{M ln(10)}{3} + i\pi k$ and $\eta_k = (-1)^{k} *\xi_k$ with $\xi_0 = \frac{1}{2}$, $\xi_k = 1$ for $1 \leq k \leq M$, $\xi_{2M} = \frac{1}{2M}$, $\xi_{2M-k} = \xi_{2M-k+1}+2^{-M}\binom{M}{k}$.
\newline

The implementation of the algorithm follows the procedure described above and can be described the following manner:
\begin{itemize}
\item{Select a size for the parameter M (for the current problem we use M as 30).}
\item{Define an array $Times$ in order to store the arguments where of the targeted function $f$.}
\item{Construct an array of factors ($\xi$) which will act as factors in the summation of $F(\frac{\beta_k}{t})$. Set $xi[0] = 1/2$. Fill the elements on positions $1$ $\to$ $M+1$ with 1s. Fill the rest of the array with the factors of the binomial expansion of order $2M$, as presented above.}
\item{Precompute the constants described by the algorithm.}
\item{For each point of time $t$, compute the sum described by the algorithm. This part of the code operates in an independent way (since we compute the sum for each point) and is suitable for a parallel implementation.}
\item{Output the vector $f_st$ containing the results of the evaluations.}
\end{itemize} 


\subsection{Abate Dubner Algorithm}
Talbot method is an alternative to Euler.







