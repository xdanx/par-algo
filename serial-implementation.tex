Some time was spent nvestigating the $Laplace$ $Transform$. Directly, the transform can be given in the following way:
$F(s) =\int_0^\infty \! f(t)e^{-st} \, \mathrm{d}t$.

Since we have established that we need to find the inverse transform, we notice the fact that:
$f(t) =\int_0^\infty \! F(s)e^{st} \, \mathrm{d}t$. Oh Baby ...
\newline

We are facing the following problem:
\begin{itemize}
\item{we do not know the analytical expression of the function $F$. }
\item{the $mistery.o$ file only allows for evaluation of the function at some points passed as inputs. }
\end{itemize}

Therefore, we have realised that we need to perform numerical inverse of $Inverse$ $Laplace$ $transform$.
The paper offered us a valuable introduction to the problem that we were facing.
The $EULER$ algorithm was a starting point, and we proceed to the implementation step:

\subsection{EULER Algorithm}
$EULER$ Algorithm can be summarised by the following formula:
$f(t) = \frac{e^{A/2}}{2t} \sum\limits_{k=-\infty}^\infty {(-1)^k Re F(\frac{A+2k\pi i}{2t} }$

The algorithm, introduced by the $Abate$ in 1968\cite{abate}. 
The input of the algorithm consists in a series of parameters.
In our implementation, we provide an array of points $t$ where the 
function $f$ must be evaluated. The result of the evaluation is returned
in the array $f_t$.

We select the input points of $t$ in the following way: given the left and
an upper bound, choose a number of points and divide the interval 
to the number of points. 

The $EULER$ algorithm makes use of the binomial coefficients, and in order 
to calculate them, we pass the order($m$) of the binomial as a parameter.
Also, since the inversion of the $Laplace$ transform requires the computation
of an integral in the continuous case, we have to compute a sum for the discrete
case. The parameter $n$ of the function call will give the number of elements under the
$\sum$ symbol.

Using the parameter $A$, as shown in \cite{abate}, one can vary the precision of the approximation.



