%% Place legend below x-axis
\makeatletter
\pgfplotsset{
    every axis x label/.append style={
        alias=current axis xlabel
    },
    legend pos/outer south/.style={
        /pgfplots/legend style={
            at={%
                (%
                \@ifundefined{pgf@sh@ns@current axis xlabel}%
                {xticklabel cs:0.5}%
                {current axis xlabel.south}%
                )%
            },
            anchor=north
        }
    }
}
\makeatother




% Initialise
\newcommand{\xlabel}{\@empty}
\newcommand{\ylabel}{\@empty}

%xlabel, ylabel, y expression (may use \numprocs)
\newcommand{\createplot}[3]{

	% Variables
	\renewcommand{\xlabel}{#1}
	\renewcommand{\ylabel}{#2}


	%%%% Create plot
	\begin{tikzpicture}
	\begin{axis}[
	xmin = 0,
	ymin = 0,
	xmode=log,
	xlabel = \xlabel,
	ylabel = \ylabel,
	% legend
	legend pos=outer south,
	legend columns=3,
	]


	% Initialise variables
	\newcommand{\plotme}[2] {
		% ##1 = color
		% ##2 = num processors
		\def \numprocs {##2}

		\addplot
		% [
		% color = ##1,
		% mark = x,
		% only marks,
		% smooth  % draw smooth curve
		% ]
		table [
		#3
		] {timings.csv};

		\addlegendentry{\textbf{##2 procs}} ;
	}

	% Lines to plot
	\plotme{red}{1}
	\plotme{blue}{2}
	\plotme{green}{4}
	\plotme{black}{8}
	\plotme{black}{16}
	\plotme{black}{22}

	\end{axis}
	\end{tikzpicture}
}




% Plot graphs


\begin{figure}[h]
    \centering
\createplot{Work}{Runtime (s)}{y = \numprocs-proc}
%\createplot{Work}{Normalised runtime (\%)}{y expr= 100 * \thisrow{\numprocs-proc} / \thisrow{1-proc}}
    \caption{Comparison of runtime for different number of processors}
    \label{fig:plot-runtime}
\end{figure}



\begin{figure}[h]
    \centering
	\createplot{Work}{Speedup (\%)}{y expr= 100 * \thisrow{1-proc} / \thisrow{\numprocs-proc}}
    \caption{Comparison of speedup and efficiency for different number of processors}
    \label{fig:plot-speedup}
\end{figure}


\begin{figure}[h]
    \centering
	\createplot{Work}{Efficiency (\%)}{y expr= 100 * \thisrow{1-proc} / ( \numprocs * \thisrow{\numprocs-proc} )}
    \caption{Comparison of speedup and efficiency for different number of processors}
    \label{fig:plot-efficiency}
\end{figure}


% Table
\begin{figure}[h]
    \centering

	\pgfplotstabletypesetfile[
		% Header names
		columns/work/.style={column name=\textsc{Work}},
		columns/1-proc/.style={column name=\textsc{1}},
		columns/2-proc/.style={column name=\textsc{2}},
		columns/4-proc/.style={column name=\textsc{4}},
		columns/8-proc/.style={column name=\textsc{8}},
		columns/16-proc/.style={column name=\textsc{16}},
		columns/22-proc/.style={column name=\textsc{22}},
		% Lines
		every head row/.style={
		before row={%
		\toprule
		 & \multicolumn{4}{c}{Number of processors}\\
		},
		after row=\midrule,
		},
		every last row/.style={after row=\bottomrule}
	]{timings.csv}

    \caption{Measured runtimes (in seconds) for different number of processors}
    \label{fig:table-runtime}
\end{figure}


